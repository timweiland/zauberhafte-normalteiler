\documentclass[a4paper]{article}

\usepackage[utf8]{inputenc}
\usepackage[T1]{fontenc}
\usepackage{textcomp}
\usepackage[german]{babel}
\usepackage{amsmath, amssymb, amsthm}
\usepackage{thmtools}
\usepackage{graphicx}
\usepackage{fancyhdr}
\usepackage[a4paper,left=3cm,right=3cm,top=3cm,
bottom=4cm,bindingoffset=5mm]{geometry}


% figure support
\usepackage{import}
\usepackage{xifthen}
\pdfminorversion=7
\usepackage{pdfpages}
\usepackage{transparent}
\usepackage{colonequals}
\newcommand{\incfig}[1]{%
    \def\svgwidth{\columnwidth}
    \import{./figures/}{#1.pdf_tex}
}

\usepackage{setspace}
\setstretch{1.5}

\declaretheorem[style=plain,numbered=no, shaded={rulecolor=Black,rulewidth=0.5pt,margin=0.5em,bgcolor={rgb}{1,1,1}}]{satz}
\declaretheorem[style=plain,numberlike=satz]{proposition}
\declaretheorem[style=plain,numbered=no]{lemma}
\declaretheorem[style=plain,numberlike=satz]{korollar}

\declaretheorem[style=definition,numbered=no]{definition}
\declaretheorem[style=definition,numberlike=satz]{beispiel}
\declaretheorem[style=definition,numbered=no]{bemerkung}
\declaretheorem[style=definition,numbered=no]{erinnerung} \declaretheoremstyle[ spaceabove=-6pt,
spacebelow=6pt,
headfont=\normalfont\bfseries,
bodyfont = \normalfont,
postheadspace=1em,
qed=$\blacksquare$,
headpunct={:}]{myproofstyle} %<---- change this name
\declaretheorem[name={Beweis}, style=myproofstyle, unnumbered]{beweis}

\fancyhf{} 
\cfoot{\thepage} 
\pagestyle{fancy} 
\fancyhead[R]{Tim Weiland, Luc Mercatoris} 
\fancyhead[L]{28. November 2019} 
\fancyhead[C]{Zauberhafte Normalteiler}


\newcommand{\R}{\mathbb{R}}
\newcommand{\Q}{\mathbb{Q}}
\newcommand{\N}{\mathbb{N}}
\newcommand{\Z}{\mathbb{Z}}
\newcommand{\O}{\varnothing}
\newcommand\logeq{\mathrel{\raisebox{.66pt}{:}}\Leftrightarrow}

\renewcommand{\labelenumi}{\roman{enumi})}

\pdfsuppresswarningpagegroup=1

\begin{document}
\thispagestyle{ErsteSeite} 
\begin{center}
    \huge{\fontfamily{put}\selectfont Zauberhafte Normalteiler}
\end{center}
\hspace{10mm}
\begin{definition}[1.1]
        Sei $(G, \circ)$ eine Gruppe und $U \subset G$ eine Untergruppe. \\
        U heißt \textbf{Normalteiler} von G $\logeq \forall g \in G: g U g^{-1} = U $
\end{definition}

\begin{bemerkung}
Im Allgemeinen ist $U \subset G$ kein Normalteiler. Wir suchen größte Untergruppe von $G$, in der $U$ Normalteiler ist. Wir wollen also $V$ finden sodass:
\begin{enumerate}
        \item $U \subset V$ ($U$ ist in $V$ enthalten)
        \item $U$ ist Normalteiler in $V$
        \item Ist $V'$ eine weitere Untergruppe von $G$ die i) und ii) erfüllt, so gilt $V' \subset V$. ($V$ ist größtmöglich)
    \end{enumerate}
\end{bemerkung}

\begin{definition}[2.2]
        Sei $(G, \circ)$ eine Gruppe und $U \subset G$ eine Untergruppe. Der \textbf{Normalisator} von $U$ in $G$ ist definiert als 
          $N_G(U) := \{g \in G  \mid g U g^{-1} = U\}$.
\end{definition}
\begin{satz}[2.3]
        $N_G(U)$ ist Untergruppe von G und erfüllt die Eigenschaften i) bis iii).
    \end{satz}

\begin{satz}[2.4]
        Seien $G, U$ und $N_G(U)$ wie gehabt. \\
        Seien ferner $m \in \N$, $x_1, \ldots, x_m \in N_G(U)$ und $u_0, \ldots, u_m \in U$. Dann gilt für ein geeignetes $u \in U$:
        \[
            u_m \circ x_m \circ u_{m-1} \circ x_{m-1} \circ \ldots \circ u_1 \circ x_1 \circ u_0 = x_m \circ \ldots \circ  x_1 \circ u
        .\] 
        Insbesondere: \\
        $x_m \circ \ldots \circ x_1 \in U \implies u_m \circ x_m \circ u_{m-1} \circ x_{m-1} \circ \ldots \circ u_1 \circ x_1 \circ u_0 \in U$.
    \end{satz}
    
    \begin{definition}[3.1]
    Die Gruppe aller Permutationen mit $n$ Karten entspricht der \textbf{symmetrischen Gruppe} $S_n$. Jede Permutation lässt sich mit $\Z_n := \{0, \ldots, n-1\}$ als bijektive Abbildung $\phi : \Z_n &\to \Z_n$ identifizieren.
    \end{definition}
    
    \begin{definition}[3.2]
        Sei $r \in Z_n$. Die Permutation $s_r  : \Z_n &\longrightarrow \Z_n, \quad k  &\longmapsto s_r(k) = k + r \mod n$ entspricht einem zyklischen Shift im Kartenstapel. $s_r$ ist bijektiv, also $s_r \in S_n$.
    \end{definition}
    
     \begin{satz}[3.4]
        $S := \{s_r  \mid r \in \Z_n\}$ ist kommutative Untergruppe von $S_n$.
    \end{satz}
    
    \begin{definition}[4.1]
    Wir definieren $\phi_a: \Z_n &\longrightarrow \Z_n, \quad
            k &\longmapsto \phi_a(k) = ak \mod n$. \\
    $\phi_a$ ist bijektiv $\Leftrightarrow$ $a$ und $n$ teilerfremd. In diesem Fall gilt $\phi_a \in S_n$.
    \end{definition}
    
    \begin{definition}[4.2]
        Seien $a, b \in \Z.$ Definiere 
            $\phi_{a,b}: \Z_n &\longrightarrow \Z_n, \quad
            k &\longmapsto \phi_{a,b}(k) = ak + b \mod n$. \\
            $\phi_{a,b}$ ist bijektiv $\Leftrightarrow a$ und $n$ teilerfremd. In diesem Fall gilt $\phi_{a, b} \in S_n$.
    \end{definition}
    
    \begin{lemma}[4.4]
        Sei $\phi \in S_n$ bijektiv. Dann gilt: \\
        $\exists a \in \Z_n^{*}, b \in \Z_n : \phi = \phi_{a, b}
        \iff \exists a \in \Z_n \; \forall k \in \Z_n: \phi(k+1) = \phi(k)+a \mod n$.
    \end{lemma}
    
    \begin{satz}[4.5]
            Der Normalisator von $S$ in $S_n$ ist $N_{S_n}(S) = \{\phi_{a, b}  \mid a, b \in \N \text{ teilerfremd}\}$.
       
    \end{satz}
    \begin{bemerkung}[5.1] 
    Seien $c,n \in \mathbb{N}$. Wir definieren folgende Mischoperationen: 
    \begin{enumerate}
     \item $R_c$: Teile die $n$ Karten von links nach rechts auf $c$ Stapel aus. Danach werden die Karten wieder aufgenommen, und zwar von rechts nach links. Ganz nach oben kommt also der Stapel ganz links.
    \item $L_c$: Von links nach rechts austeilen, dieses Mal von links nach rechts aufnehmen.  Ganz nach oben kommt also der Stapel ganz rechts.
    \end{enumerate}
    \end{bemerkung}   
    
    \begin{satz}[5.2]
    $R_c$ und $L_c$ wie gehabt. Dann gilt:
    \begin{enumerate}
    \item Wenn $c$ ein Teiler von $n - 1$ und $a := (n - 1)/c$, dann sind $a$ und $n$ teilerfremd und $R_c$ entspricht $\phi_{a,a}$.
    \item Wenn $c$ ein Teiler von $n + 1$ und $a := (n + 1)/c$, dann sind $a$ und $n$ teilerfremd und $L_c$ entspricht $\phi_{-a,-1}$.
    \end{enumerate}
    \end{satz} \\
    
    \begin{lemma}[5.3]
    Für $n \in $ \mathbb{N} sei $A_n := \lbrace x  \mid  n-1$ durch $x$ teilbar$\rbrace$ und $B_n := \lbrace x  \mid  n+1$ durch $x$ teilbar$\rbrace$. \\
    Wähle $a_1, …. a_r \in A_n$ und $a_1', …, a_l' \in B_n$ aus. Sei $a$ das Produkt aller $a_i$ und $a_j'$. Wähle $c_i$ bzw $c_j'$ sodass $c_ia_i = n-1$, bzw $c_j'a_j' = n+1$. \\
    Führe auf einen Kartenstapel $R_{c_1}, …, R_{c_r}$ und $L_{c_1'}, …, L_{c_l'}$ in beliebiger Reihenfolge durch. Dazwischen darf noch zusätzlich abgehoben werden. Der Kartenstapel befindet sich in der Permutation $\phi_{a,b}$, mit unbekanntem $b$ falls das Abheben beliebig stattfindet.
    \end{lemma}

   \noindent \fbox{\parbox{\textwidth} {Für uns von zentraler Bedeutung: $a=1$ oder $a=-1$\\ 
   \begin{Large}$\Longrightarrow$ \end{Large} \textbf{Die Karten sind in der gleichen, bzw der gespiegelten zyklischen Reihenfolge.}}} \\
   
   \noindent \underline{Im Fall vom Originaltrick:} \\
   $n=9, \, A_n=\lbrace 1,2,4,8\rbrace$. Wir wählen $a_1 = a_2 = a_3 = 4$ und somit $c_1 = c_2 = c_3 = 2$. \\
   Es gilt $a = a_1 \cdot a_2 \cdot a_3 = 4 \cdot 4 \cdot 4 = 64 \equiv 1$ mod $9$. Wendet man $R_2$ also 3 Mal auf den Kartenstapel an, so ist das äquivalent zu $\phi_{1,r} = s_r$. Das ganze entspricht also einem zyklischen Shift um $r$ Stellen.
   
\end{document}
