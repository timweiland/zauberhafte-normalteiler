\documentclass[a4paper]{article}

\usepackage[utf8]{inputenc}
\usepackage[T1]{fontenc}
\usepackage{textcomp}
\usepackage[german]{babel}
\usepackage{amsmath, amssymb, amsthm}
\usepackage{thmtools}
\usepackage{graphicx}
\usepackage{fancyhdr}
\usepackage[a4paper,left=3cm,right=3cm,top=3cm,
bottom=4cm,bindingoffset=5mm]{geometry}


% figure support
\usepackage{import}
\usepackage{xifthen}
\pdfminorversion=7
\usepackage{pdfpages}
\usepackage{transparent}
\usepackage{colonequals}
\newcommand{\incfig}[1]{%
    \def\svgwidth{\columnwidth}
    \import{./figures/}{#1.pdf_tex}
}

\usepackage{setspace}
\setstretch{1.5}

\declaretheorem[style=plain,numbered=no, shaded={rulecolor=Black,rulewidth=0.5pt,margin=0.5em,bgcolor={rgb}{1,1,1}}]{satz}
\declaretheorem[style=plain,numberlike=satz]{proposition}
\declaretheorem[style=plain,numberlike=satz]{lemma}
\declaretheorem[style=plain,numberlike=satz]{korollar}

\declaretheorem[style=definition,numbered=no]{definition}
\declaretheorem[style=definition,numberlike=satz]{beispiel}
\declaretheorem[style=definition,numbered=no]{bemerkung}
\declaretheorem[style=definition,numbered=no]{erinnerung} \declaretheoremstyle[ spaceabove=-6pt,
spacebelow=6pt,
headfont=\normalfont\bfseries,
bodyfont = \normalfont,
postheadspace=1em,
qed=$\blacksquare$,
headpunct={:}]{myproofstyle} %<---- change this name
\declaretheorem[name={Beweis}, style=myproofstyle, unnumbered]{beweis}

\pagestyle{fancy} 
\fancyhf{} 
\fancyhead[L]{Text 1} 
\fancyhead[C]{Text 2} 
\fancyfoot[R]{\thepage} 

\fancypagestyle{ErsteSeite}{% 
   \fancyhf{}% 
   \fancyhead[R]{Weiland Tim, Mercatoris Luc} 
   \fancyhead[L]{28. November 2019} 
} 

\newcommand{\R}{\mathbb{R}}
\newcommand{\Q}{\mathbb{Q}}
\newcommand{\N}{\mathbb{N}}
\newcommand{\Z}{\mathbb{Z}}
\newcommand{\O}{\varnothing}
\newcommand*{\longeq}{\ratio\Longleftrightarrow}

\renewcommand{\labelenumi}{\roman{enumi})}

\pdfsuppresswarningpagegroup=1

\begin{document}
\thispagestyle{ErsteSeite} 
\begin{center}
\framebox{\huge{Zauberhafte Normalteiler}}
\end{center}
\hspace{10mm}
\begin{definition}[1.1]
        Sei $(G, \circ)$ eine Gruppe und $U \subset G$ eine Untergruppe. \\
        U heißt \textbf{Normalteiler} von G $\iff \forall g \in G: g U g^{-1} = U $
\end{definition}

\begin{bemerkung}
Im Allgemeinen ist $U \subset G$ kein Normalteiler Wir suchen größte Untergruppe von $G$, in der $U$ Normalteiler ist. Wir wollen also $V$ finden sodass
\begin{enumerate}
        \item $U \subset V$ ($U$ ist in $V$ enthalten)
        \item $U$ ist Normalteiler in $V$
        \item Ist $V'$ eine weitere Untergruppe von $G$ die i) und ii) erfüllt, so gilt $V' \subset V$. ($V$ ist größtmöglich)
    \end{enumerate}
\end{bemerkung}

\begin{definition}[2.2]
        Der \textbf{Normalisator} von $U$ in $G$ ist definiert als 
          $N_G(U) := \{g \in G  \mid g U g^{-1} = U\}$
\end{definition}
\begin{satz}[2.3]
        $N_G(U)$ ist Untergruppe von G und erfüllt die Eigenschaften i) bis iii).
    \end{satz}

\begin{satz}[2.4]
        Seien $G, U$ und $N_G(U)$ wie gehabt. \\
        Seien ferner $m \in \N$, $x_1, \ldots, x_m \in N_G(U)$ und $u_0, \ldots, u_m \in U$. Dann gilt für ein geeignetes $u \in U$:
        \[
            u_m \circ x_m \circ u_{m-1} \circ x_{m-1} \circ \ldots \circ u_1 \circ x_1 \circ u_0 = x_m \circ \ldots \circ  x_1 \circ u
        .\] 
        Insbesondere: \\
        $x_m \circ \ldots \circ x_1 \in U \implies u_m \circ x_m \circ u_{m-1} \circ x_{m-1} \circ \ldots \circ u_1 \circ x_1 \circ u_0 \in U$.
    \end{satz}
    
    \begin{definition}[3.1]
    Die Gruppe aller Permutationen mit $n$ Karten heißt die \textbf{symmetrische Gruppe} $S_n$. Jede Permutation lässt sich mit $\Z_n := \{0, \ldots, n-1\}$ als bijektive Abbildung $\phi : \Z_n &\to \Z_n$ identifizieren.
    \end{definition}
    
    \begin{definition}[3.2]
        Sei $r \in Z_n$. Die Permutation $s_r  : \Z_n &\longrightarrow \Z_n, \quad k  &\longmapsto s_r(k) = k + r \mod n$ entspricht einem zyklischen Shift im Kartenstapel. $s_r$ ist bijektiv, also $s_r \in S_n$.
    \end{definition}
    
     \begin{satz}[3.4]
        $S := \{s_r  \mid r \in \Z_n\}$
        ist kommutative Untergruppe von $S_n$. \\
        Das bedeutet insbesondere: Die Reihenfolge von mehrmaligem Abheben ist egal (Kommutativität), und mehrmaliges Abheben bringt die Karten nicht mehr durcheinander als einmaliges Abheben (Abgeschlossenheit).
    \end{satz}
\end{document}